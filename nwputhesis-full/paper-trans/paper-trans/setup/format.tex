%%%%%%%%%%%%%%%%%%%%%%%%%%%%%%%%%%%%%%%%%%%%%%%%%%%%%%%%%%%%%%%%%%%%%%%%%
%
%   LaTeX File for Doctor (Master) Thesis of Tsinghua University
%   LaTeX + CJK     清华大学博士(硕士)论文模板
%   Based on Wang Tianshu's Template for XJTU
%   Version: 1.00
%   Last Update: 2003-09-12
%
%%%%%%%%%%%%%%%%%%%%%%%%%%%%%%%%%%%%%%%%%%%%%%%%%%%%%%%%%%%%%%%%%%%%%%%%%
%   Copyright 2002-2003  by  Lei Wang (BaconChina)       (bcpub@sina.com)
%%%%%%%%%%%%%%%%%%%%%%%%%%%%%%%%%%%%%%%%%%%%%%%%%%%%%%%%%%%%%%%%%%%%%%%%%

%%%%%%%%%%%%%%%%%%%%%%%%%%%%%%%%%%%%%%%%%%%%%%%%%%%%%%%%%%%%%%%%%%%%%%%%%
%
%   LaTeX File for phd thesis of xi'an Jiao Tong University
%
%%%%%%%%%%%%%%%%%%%%%%%%%%%%%%%%%%%%%%%%%%%%%%%%%%%%%%%%%%%%%%%%%%%%%%%%%
%   Copyright 2002  by  Wang Tianshu    (tswang@asia.com)
%%%%%%%%%%%%%%%%%%%%%%%%%%%%%%%%%%%%%%%%%%%%%%%%%%%%%%%%%%%%%%%%%%%%%%%%%
%%%%%%%%%%%%%%%%%%%%%%%%%%%%%%%%%%%%%%%%%%%%%%%%%%%%%%%%%%%
%
% 主文档 格式定义
%
%%%%%%%%%%%%%%%%%%%%%%%%%%%%%%%%%%%%%%%%%%%%%%%%%%%%%%%%%%%

% 按清华标准, 将版芯控制在240mm以内, 正文范围控制在220mm以内
%\addtolength{\headsep}{-0.1cm}          %页眉位置
%\addtolength{\footskip}{-0.1cm}         %页脚位置
\addtolength{\topmargin}{0.5cm}

%%%%%%%%%%%%%%%%%%%%%%%%%%%%%%%%%%%%%%%%%%%%%%%%%%%%%%%%%%%
% 公式的精调
%%%%%%%%%%%%%%%%%%%%%%%%%%%%%%%%%%%%%%%%%%%%%%%%%%%%%%%%%%%

%\setlength{\mathindent}{4.7 em}     %左对齐公式缩进量

% \eqnarray如果很长,影响分栏、换行和分页(整块挪动,造成页面空白),
% 可以设置成为自动调整模式
\allowdisplaybreaks[4]

%%%%%%%%%%%%%%%%%%%%%%%%%%%%%%%%%%%%%%%%%%%%%%%%%%%%%%%%%%%
%下面这组命令使浮动对象的缺省值稍微宽松一点,从而防止幅度
%对象占据过多的文本页面,也可以防止在很大空白的浮动页上放置
%很小的图形。
%%%%%%%%%%%%%%%%%%%%%%%%%%%%%%%%%%%%%%%%%%%%%%%%%%%%%%%%%%%
\renewcommand{\textfraction}{0.15}
\renewcommand{\topfraction}{0.85}
\renewcommand{\bottomfraction}{0.65}
\renewcommand{\floatpagefraction}{0.60}


%%%%%%%%%%%%%%%%%%%%%%%%%%%%%%%%%%%%%%%%%%%%%%%%%%%%%%%%%%%
%下面这组命令可以使公式编号随着每开始新的一节而重新开始。
%%%%%%%%%%%%%%%%%%%%%%%%%%%%%%%%%%%%%%%%%%%%%%%%%%%%%%%%%%%

%\makeatletter      % '@' is now a normail "letter" for TeX
%\@addtoreset{eqation}{section}
%\makeatother       % '@' is restored as a "non-letter" character for TeX

%%%%%%%%%%%%%%%%%%%%%%%%%%%%%%%%%%%%%%%%%%%%%%%%%%%%%%%%%%%
% 重定义字体命令
%%%%%%%%%%%%%%%%%%%%%%%%%%%%%%%%%%%%%%%%%%%%%%%%%%%%%%%%%%%
% 注意win2000,没有 simsun, 最好到网上找一个
% 一些字体是office2000 带的
%%%%%%%%%%%%%%%%%%%%%%%%%%%%%%%%%%%%%%%%%%%%%%%%%%%%%%%%%%%

\setmainfont{TeX Gyre Termes}
%\setmainfont{Latin Modern Roman}
\setsansfont{TeX Gyre Heros}
\setCJKmainfont[BoldFont={方正小标宋简体}]{方正书宋简体}    % 宋体  
\setCJKsansfont{Adobe Heiti Std}
\setCJKmonofont{Adobe Fangsong Std}

%\setCJKfamilyfont{song}[BoldFont={方正宋黑简体}]{SimSun}      	% 宋体  
%\setCJKfamilyfont{song}[BoldFont={方正宋三_GBK}]{方正博雅宋_GBK}  % 宋体  
%\setCJKfamilyfont{song}[BoldFont={Adobe Heiti Std}]{Adobe Song Std}    % 宋体  
%\setCJKfamilyfont{song}[BoldFont={华文中宋}]{华文宋体}    % 宋体  
%\setCJKfamilyfont{song}[BoldFont={方正大标宋_GBK}]{方正兰亭宋_GBK}    % 宋体  
\setCJKfamilyfont{song}[BoldFont={方正小标宋简体}]{方正书宋简体}    % 宋体  
\setCJKfamilyfont{hei}{Adobe Heiti Std}      	% 黑体  
\setCJKfamilyfont{kai}{Adobe Kaiti Std}      	% 楷体  
\setCJKfamilyfont{fang}{Adobe Fangsong Std}  	% 仿宋体
\setCJKfamilyfont{nwpulogo}{nwpulogo}        	% 含"西北工业大学"logo字体 

\newcommand{\song}{\CJKfamily{song}}
\newcommand{\hei}{\CJKfamily{hei}}
\newcommand{\fang}{\CJKfamily{fang}}
\newcommand{\kai}{\CJKfamily{kai}}
\newcommand{\nwpulogo}{\CJKfamily{nwpulogo}}

%%%%%%%%%%%%%%%%%%%%%%%%%%%%%%%%%%%%%%%%%%%%%%%%%%%%%%%%%%%
% 重定义字号命令
%%%%%%%%%%%%%%%%%%%%%%%%%%%%%%%%%%%%%%%%%%%%%%%%%%%%%%%%%%%

\newcommand{\chuhao}{\fontsize{42pt}{63pt}\selectfont}    % 初号, 1.5倍行距
\newcommand{\yihao}{\fontsize{26pt}{36pt}\selectfont}    % 一号, 1.4倍行距
\newcommand{\erhao}{\fontsize{22pt}{28pt}\selectfont}    % 二号, 1.25倍行距
\newcommand{\xiaoer}{\fontsize{18pt}{18pt}\selectfont}    % 小二, 单倍行距
\newcommand{\sanhao}{\fontsize{16pt}{24pt}\selectfont}    % 三号, 1.5倍行距
\newcommand{\xiaosan}{\fontsize{15pt}{22pt}\selectfont}    % 小三, 1.5倍行距
\newcommand{\sihao}{\fontsize{14pt}{21pt}\selectfont}    % 四号, 1.5倍行距
\newcommand{\banxiaosi}{\fontsize{13pt}{19.5pt}\selectfont}    % 半小四, 1.5倍行距
\newcommand{\xiaosi}{\fontsize{12pt}{14.4pt}\selectfont}    % 小四, 1.25倍行距
\newcommand{\dawuhao}{\fontsize{11pt}{11pt}\selectfont}    % 大五号, 单倍行距
\newcommand{\wuhao}{\fontsize{10.5pt}{10.5pt}\selectfont}    % 五号, 单倍行距
\newcommand{\xiaowu}{\fontsize{9pt}{9pt}\selectfont}		% 小五号



%%%%%%%%%%%%%%%%%%%%%%%%%%%%%%%%%%%%%%%%%%%%%%%%%%%%%%%%%%%
% 重定义一些正文相关标题
%%%%%%%%%%%%%%%%%%%%%%%%%%%%%%%%%%%%%%%%%%%%%%%%%%%%%%%%%%%

% qiuying comment
%\theoremstyle{plain} \theorembodyfont{\song\rmfamily}
%\theoremheaderfont{\hei\rmfamily} \theoremseparator{:}
%\newtheorem{definition}{\hei 定义}[chapter]
%\newtheorem{proposition}[definition]{\hei 命题}
%\newtheorem{lemma}[definition]{\hei 引理}
%\newtheorem{theorem}{\hei 定理}[chapter]
%\newtheorem{axiom}{\hei 公理}
%\newtheorem{corollary}[definition]{\hei 推论}
%\newtheorem{exercise}[definition]{}
%
%\theoremheaderfont{\CJKfamily{hei}\rmfamily}\theorembodyfont{\rmfamily}
%\theoremstyle{nonumberplain} \theoremseparator{:}
%\theoremsymbol{$\blacksquare$}
%\newtheorem{proof}{\hei 证明}
%
%\theoremsymbol{$\square$}
%\newtheorem{example}{\hei 例}
%

%%%%%%%%%%%%%%%%%%%%%%%%%%%%%%%%%%%%%%%%%%%%%%%%%%%%%%%%%%%
% 用于中文段落缩进 和正文版式
%%%%%%%%%%%%%%%%%%%%%%%%%%%%%%%%%%%%%%%%%%%%%%%%%%%%%%%%%%%
%\CJKcaption{GB_aloft}
\xeCJKcaption{gb_452}

\newlength \CJKtwospaces

\def\CJKindent{
    \settowidth\CJKtwospaces{\CJKchar{"0A1}{"0A1}\CJKchar{"0A1}{"0A1}}%
    \parindent\CJKtwospaces
}


%\CJKtilde  \CJKindent

\renewcommand\contentsname{目~~~~录}
\renewcommand\chaptername{\CJKprechaptername\CJKthechapter\CJKchaptername}

%%%%%%%%%%%%%%%%%%%%%%%%%%%%%%%%%%%%%%%%%%%%%%%%%%
%定义段落章节的标题和目录项的格式
%%%%%%%%%%%%%%%%%%%%%%%%%%%%%%%%%%%%%%%%%%%%%%%%%%
\setcounter{secnumdepth}{4}
\setcounter{tocdepth}{4}

% Modified By Lei Wang BaconChina
% THU Version
\titleformat{\chapter}[hang]
    {\normalfont\sanhao\filcenter\hei\sf}
    {\sanhao{\chaptertitlename}}
    {20pt}{\sanhao}
%\titlespacing{\chapter}{0pt}{-3ex  plus .1ex minus .2ex}{2.5ex plus .1ex minus .2ex}
\titlespacing{\chapter}{0pt}{-3ex  plus .1ex minus .2ex}{0.25em}

\titleformat{\section}[hang]{\hei \sf \sihao}
    {\sihao \thesection}{0.5em}{}{}
%\titlespacing{\section}{0pt}{1.5ex plus .1ex minus .2ex}{\wordsep}
\titlespacing{\section}{0pt}{-0.2em}{0em}

\titleformat{\subsection}[hang]{\hei \sf \banxiaosi}
    {\banxiaosi \thesubsection}{0.5em}{}{}
%    {\banxiaosi \thesubsection}{0pt}{}{}
%\titlespacing{\subsection}{0pt}{1.5ex plus .1ex minus .2ex}{\wordsep}
\titlespacing{\subsection}{0pt}{-0.25em}{0em}

\titleformat{\subsubsection}[hang]{\hei \sf}
    {\thesubsubsection }{0.5em}{}{}
%\titlespacing{\subsubsection}{0pt}{1.2ex plus .1ex minus .2ex}{\wordsep}
\titlespacing{\subsubsection}{0pt}{0.25em}{0pt}

%去掉中间对齐的sectionformat,这样就把节的标题左对齐了。
%\renewcommand \sectionformat{}

% 按清华标准, 缩小目录中各级标题之间的缩进
\dottedcontents{chapter}[0.0em]{\hei\vspace{0.5em}}{0.0em}{5pt}
\dottedcontents{section}[1.16cm]{}{1.8em}{5pt}
\dottedcontents{subsection}[2.00cm]{}{2.7em}{5pt}
\dottedcontents{subsubsection}[2.86cm]{}{3.4em}{5pt}

%%%%%%%%%%%%%%%%%%%%%%%%%%%%%%%%%%%%%%%%%%%%%%%%%%%%%%%
% 定义页眉和页脚 使用fancyhdr 宏包
%%%%%%%%%%%%%%%%%%%%%%%%%%%%%%%%%%%%%%%%%%%%%%%%%%%%%%%%

\newcommand{\makeheadrule}{%
    \makebox[0pt][l]{\rule[.7\baselineskip]{\headwidth}{0.8pt}}%
% 1 Line Modified by Lei Wang BaconChina
% XJTU Version
%    \rule[.6\baselineskip]{\headwidth}{0.4pt}\vskip-.8\baselineskip}
% THU Version
    \vskip-.8\baselineskip}

\makeatletter
\renewcommand{\headrule}{%
    {\if@fancyplain\let\headrulewidth\plainheadrulewidth\fi
     \makeheadrule}}

\pagestyle{fancyplain}

%去掉章节标题中的数字
%\renewcommand{\chaptermark}[1]{\markboth{\chaptername \ #1}{}}

 \fancyhf{}
% \fancyfoot[C,C]{\thepage}

%在book文件类别下,\leftmark自动存录各章之章名,\rightmark记录节标题

% Modified by Lei Wang BaconChina
% XJTU Version
% \fancyhead[RO]{\CJKfamily{song}\leftmark}
% \fancyhead[LE]{\CJKfamily{song}西安交通大学博士学位论文}
% \fancyfoot[C,C]{--~\thepage~--}
% THU Version
% \fancyhead[CO]{\CJKfamily{song}\wuhao\leftmark}
% \fancyhead[CE]{\nwpulogo\fontsize{8pt}{6pt} 西北工业大学~~~ \sanhao\song 本科毕业设计论文}
 \fancyfoot[C,C]{\wuhao-~\thepage~-}
\chead{\sanhao\raisebox{0.04cm}{\nwpulogo 西北工业大学} \song \bfseries{本科毕业设计英文翻译}}

%%%%%%%%%%%%%%%%%%%%%%%%%%%%%%%%%%%%%%%%%%%%%%%%%%%%%%%%
% 设置行距和段落间垂直距离
%%%%%%%%%%%%%%%%%%%%%%%%%%%%%%%%%%%%%%%%%%%%%%%%%%%%%%%%

% 段落之间的竖直距离
\setlength{\parskip}{3pt plus1pt minus1pt}

% 定义行距
\renewcommand{\baselinestretch}{1.25}

%%%%%%%%%%%%%%%%%%%%%%%%%%%%%%%%%%%%%%%%%%%%%%%%%%%%%%%%
% 调整列表环境的垂直间距
%%%%%%%%%%%%%%%%%%%%%%%%%%%%%%%%%%%%%%%%%%%%%%%%%%%%%%%%
\let\orig@Itemize =\itemize
\let\orig@Enumerate =\enumerate
\let\orig@Description =\description

\def\Myspacing{\itemsep=1.5ex \topsep=-0.5ex \partopsep=0pt \parskip=0pt \parsep=0.5ex}

\def\newitemsep{
\renewenvironment{itemize}{\orig@Itemize\Myspacing}{\endlist}
\renewenvironment{enumerate}{\orig@Enumerate\Myspacing}{\endlist}
\renewenvironment{description}{\orig@Description\Myspacing}{\endlist}
}

\def\olditemsep{
\renewenvironment{itemize}{\orig@Itemize}{\endlist}
\renewenvironment{enumerate}{\orig@Enumerate}{\endlist}
\renewenvironment{description}{\orig@Description}{\endlist}
}

\newitemsep

%%%%%%%%%%%%%%%%%%%%%%%%%%%%%%%%%%%%%%%%%%%%%%%%%%%%%%%
% 修改引用的格式,
%%%%%%%%%%%%%%%%%%%%%%%%%%%%%%%%%%%%%%%%%%%%%%%%%%%%%%%

%第一行在引用处数字两边加方框
%第二行去除参考文献里数字两边的方框
%\makeatletter
%\def\@cite#1{\mbox{$\m@th^{\hbox{\@ove@rcfont[#1]}}$}}
%\renewcommand\@biblabel[1]{#1}
%\makeatother

% 增加 \ucite 命令使显示的引用为上标形式
\newcommand{\ucite}[1]{$^{\mbox{\scriptsize \cite{#1}}}$}

%%%%%%%%%%%%%%%%%%%%%%%%%%%%%%%%%%%%%%%%%%%%%%%%%%%%%%%%%%%
%
% 定制浮动图形和表格标题样式
%
%%%%%%%%%%%%%%%%%%%%%%%%%%%%%%%%%%%%%%%%%%%%%%%%%%%%%%%%%%%

\renewcommand{\captionfont}{\CJKfamily{song}\rmfamily}
\renewcommand{\captionlabelfont}{\CJKfamily{song}\rmfamily}

% 按清华标准, 去掉图表号后面的:
\renewcommand{\captionlabeldelim}{\hspace{1em}}

% 按清华标准, 图表标题字体为11pt, 这里写作大五号
\renewcommand{\captionfont}{\dawuhao}

%%%%%%%%%%%%%%%%%%%%%%%%%%%%%%%%%%%%%%%%%%%%%%%%%%%%%%%
% 定义题头格言的格式
%%%%%%%%%%%%%%%%%%%%%%%%%%%%%%%%%%%%%%%%%%%%%%%%%%%%%%%

%
% 用法 \begin{Aphorism}{author}
%         aphorism
%      \end{Aphorism}

\newsavebox{\AphorismAuthor}
\newenvironment{Aphorism}[1]
{\vspace{0.5cm}\begin{sloppypar} \slshape
\sbox{\AphorismAuthor}{#1}
\begin{quote}\small\itshape }
{\\ \hspace*{\fill}------\hspace{0.2cm} \usebox{\AphorismAuthor}
\end{quote}
\end{sloppypar}\vspace{0.5cm}}

%自定义一个空命令,用于注释掉文本中不需要的部分。
\newcommand{\comment}[1]{}

% This is the flag for longer version
\newcommand{\longer}[2]{#1}

\newcommand{\ds}{\displaystyle}

% define graph scale
\def\gs{1.0}

%%%%%%%%%%%%%%%%%%%%%%%%%%%%%%%%%%%%%%%%%%%%%%%%%%%%%%%%%%%%%%%%%%%%%%
% 自定义项目列表标签及格式 \begin{denselist} 列表项 \end{denselist}
%%%%%%%%%%%%%%%%%%%%%%%%%%%%%%%%%%%%%%%%%%%%%%%%%%%%%%%%%%%%%%%%%%%%%%
\newcounter{newlist} %自定义新计数器
\newenvironment{denselist}[1][可改变的列表题目]{%%%%%定义新环境
\begin{list}{\textbf{\hei #1} \arabic{newlist}:} %%标签格式
    {
    \usecounter{newlist}
     \setlength{\labelwidth}{22pt} %标签盒子宽度
     \setlength{\labelsep}{0cm} %标签与列表文本距离
     \setlength{\leftmargin}{0cm} %左右边界
     \setlength{\rightmargin}{0cm}
     \setlength{\parsep}{0ex} %段落间距
     \setlength{\itemsep}{0ex} %标签间距
     \setlength{\itemindent}{44pt} %标签缩进量
     \setlength{\listparindent}{22pt} %段落缩进量
    }}
{\end{list}}%%%%%

%添加一些有用的命令
%Chinese style for the chapter reference. It doesn't work with hyperref
\newcommand{\chref}[1]{\CJKnumber{\ref{#1}}}
%adjust Chinese parenthesis space
\newcommand{\KH}[1]{\!\!(#1)\!\!}
\newcommand\dlmu@underline[2][5cm]{\hskip1pt\underline{\hb@xt@ #1{\hss#2\hss}}\hskip3pt}
\let\coverunderline\dlmu@underline

\setlength{\parindent}{2em}
\setlength{\headheight}{24pt}

