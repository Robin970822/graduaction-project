%%%%%%%%%%%%%%%%%%%%%%%%%%%%%%%%%%%%%%%%%%%%%%%%%%%%%%%%%%%%%%%%%%%%%%%%%
%
%   LaTeX File for Doctor (Master) Thesis of Tsinghua University
%   LaTeX + CJK     清华大学博士\KH{硕士}论文模板
%   Based on Wang Tianshu's Template for XJTU
%   Version: 1.00
%   Last Update: 2003-09-12
%
%%%%%%%%%%%%%%%%%%%%%%%%%%%%%%%%%%%%%%%%%%%%%%%%%%%%%%%%%%%%%%%%%%%%%%%%%
%   Copyright 2002-2003  by  Lei Wang (BaconChina)       (bcpub@sina.com)
%%%%%%%%%%%%%%%%%%%%%%%%%%%%%%%%%%%%%%%%%%%%%%%%%%%%%%%%%%%%%%%%%%%%%%%%%


%%%%%%%%%%%%%%%%%%%%%%%%%%%%%%%%%%%%%%%%%%%%%%%%%%%%%%%%%%%%%%%%%%%%%%%%%
%
%   LaTeX File for phd thesis of xi'an Jiao Tong University
%
%%%%%%%%%%%%%%%%%%%%%%%%%%%%%%%%%%%%%%%%%%%%%%%%%%%%%%%%%%%%%%%%%%%%%%%%%
%   Copyright 2002  by  Wang Tianshu    (tswang@asia.com)
%%%%%%%%%%%%%%%%%%%%%%%%%%%%%%%%%%%%%%%%%%%%%%%%%%%%%%%%%%%%%%%%%%%%%%%%%
\renewcommand{\baselinestretch}{1.5}
\fontsize{12pt}{13pt}\selectfont

\chapter*{摘~~~~要}
\markboth{中~文~摘~要}{中~文~摘~要}
强化学习是最近倍受关注的学习模型。强化学习通过智能体与环境的交互进行学习,以改变智能体的动作响应以获得适应环境最优的行为策略。深度强化学习通过应用深度神经网络模型,提升了强化学习对于高维复杂问题上的性能,已经广泛应用于机器人领域。然而在真实环境中,对于轮式机器人的复杂控制任务和决策系统,深度强化学习往往需要大量的额外工作以保证数据有效性和样本利用效率。

本文对轮式机器人决策系统和强化学习算法做了简要的介绍,接着对于基于值函数和策略搜索的深度强化学习,描述了它们的基本原理。同时详细阐述了基于值函数的DQN和基于演员-评论家模型的DDPG的实现和训练过程。此外,本文提出了应用多智能体异步训练的方法,以解决深度强化学习在实际应用中面临的轮式机器人时延问题和采样速度问题。

本文基于机器人操作系统ROS和仿真平台Gazebo搭建了轮式机器人物理仿真环境,并在该环境内进行了DQN、DDPG以及多智能体异步DDPG的训练和测试,验证了它们的性能和效果。在真实环境下,通过使用上述算法的轮式机器人智能体与人工设计的确定型有限状态机的敌方轮式机器人进行对战,探究了上述算法在真实环境中的适应能力。

\vspace{1em}
\noindent {\hei 关键词:} \quad 强化学习,ROS,轮式机器人,决策系统

