%%%%%%%%%%%%%%%%%%%%%%%%%%%%%%%%%%%%%%%%%%%%%%%%%%%%%%%%%%%%%%%%%%%%%%%%%
%
%   LaTeX File for Doctor (Master) Thesis of Tsinghua University
%   LaTeX + CJK     清华大学博士\KH{硕士}论文模板
%   Based on Wang Tianshu's Template for XJTU
%   Version: 1.00
%   Last Update: 2003-09-12
%
%%%%%%%%%%%%%%%%%%%%%%%%%%%%%%%%%%%%%%%%%%%%%%%%%%%%%%%%%%%%%%%%%%%%%%%%%
%   Copyright 2002-2003  by  Lei Wang (BaconChina)       (bcpub@sina.com)
%%%%%%%%%%%%%%%%%%%%%%%%%%%%%%%%%%%%%%%%%%%%%%%%%%%%%%%%%%%%%%%%%%%%%%%%%

%%%%%%%%%%%%%%%%%%%%%%%%%%%%%%%%%%%%%%%%%%%%%%%%%%%%%%%%%%%%%%%%%%%%%%%%%
%
%   LaTeX File for phd thesis of xi'an Jiao Tong University
%
%%%%%%%%%%%%%%%%%%%%%%%%%%%%%%%%%%%%%%%%%%%%%%%%%%%%%%%%%%%%%%%%%%%%%%%%%
%   Copyright 2002  by  Wang Tianshu    (tswang@asia.com)
%%%%%%%%%%%%%%%%%%%%%%%%%%%%%%%%%%%%%%%%%%%%%%%%%%%%%%%%%%%%%%%%%%%%%%%%%

%%%%%%%%%%%%%%%%%%%%%%%%%%%%%%%%%%%%%%%%%%%%%%%%%%%%%%%%%%%
%
% Latex 西安交通大学博士论文的模板.
%
% 建议使用miktex2.1最大安装编译此模板
%
%%%%%%%%%%%%%%%%%%%%%%%%%%%%%%%%%%%%%%%%%%%%%%%%%%%%%%%%%%


%draft 选项可以使插入的图形只显示外框,以加快预览速度。
%fleqn 让公式左对齐。
\documentclass[12pt,a4paper,openany,oneside]{book}
%\documentclass[11pt,a4paper,openany,draft]{book}
%\documentclass[11pt,a4paper,fleqn,openany,draft]{book}
%\documentclass[11pt,a4paper,fleqn,openany,draft]{book}

%以下是采用dvipdfmx所需设置
%\AtBeginDvi{\special{pdf:tounicode GBK-EUC-UCS2}}
%\usepackage[CJKbookmarks=true,dvipdfm,
%           hyperindex=true,
%           pdfstartview=FitH,
%           bookmarksnumbered=true,
%           bookmarksopen=true,
%           colorlinks=true, %注释掉此项则交叉引用为彩色边框(将colorlinks和pdfborder同时注释掉)
%           pdfborder=001,   %注释掉此项则交叉引用为彩色边框
%           citecolor=blue%
%           ]{hyperref}
%%%%%%%%%%%%%%%%%%%%%%%%%%%%%%%%%%%%%%%%%%%%%%%%%%%%%%%%%%%
%
% 引用的宏包
%
%%%%%%%%%%%%%%%%%%%%%%%%%%%%%%%%%%%%%%%%%%%%%%%%%%%%%%%%%%%

\input{setup/package.tex}

\begin{document}

%定义所有的eps文件在 figures 子目录下
\graphicspath{{figures/}}

%%%%%%%%%%%%%%%%%%%%%%%%%%%%%%%%%%%%%%%%%%%%%%%%%%%%%%%%%%%
%
%  文本格式定义
%
%%%%%%%%%%%%%%%%%%%%%%%%%%%%%%%%%%%%%%%%%%%%%%%%%%%%%%%%%%%

\input{setup/format.tex}

%%%%%%%%%%%%%%%%%%%%%%%%%%%%%%%%%%%%%%%%%%%%%%%%%%%%%%%%%%%
%
% 正文部分
%
%%%%%%%%%%%%%%%%%%%%%%%%%%%%%%%%%%%%%%%%%%%%%%%%%%%%%%%%%%%

%--- Preface ------------------------
\frontmatter

% 解决中英文混排的断行问题,会加入间距,但不会影响断行
\sloppy

\pagenumbering{Roman}

%封面
%%%%%%%%%%%%%%%%%%%%%%%%%%%%%%%%%%%%%%%%%%%%%%%%%%%%%%%%%%%%%%%%%%%%%%%%%
%
%   LaTeX File for Doctor (Master) Thesis of Tsinghua University
%   LaTeX + CJK     清华大学博士(硕士)论文模板
%   Based on Wang Tianshu's Template for XJTU
%   Version: 1.00
%   Last Update: 2003-09-12
%
%%%%%%%%%%%%%%%%%%%%%%%%%%%%%%%%%%%%%%%%%%%%%%%%%%%%%%%%%%%%%%%%%%%%%%%%%
%   Copyright 2002-2003  by  Lei Wang (BaconChina)       (bcpub@sina.com)
%%%%%%%%%%%%%%%%%%%%%%%%%%%%%%%%%%%%%%%%%%%%%%%%%%%%%%%%%%%%%%%%%%%%%%%%%

%%%%%%%%%%%%%%%%%%%%%%%%%%%%%%%%%%%%
% 封一
%%%%%%%%%%%%%%%%%%%%%%%%%%%%%%%%%%%%

\begin{titlepage}
\voffset 2.7cm
\begin{center}
\begin{center}
\begin{minipage}[c]{2.64cm}
\centering
\resizebox{!}{0.9cm}{%
\parbox{0.54cm}{\input{logo}}
}
\end{minipage}
\hskip 0.8cm
\begin{minipage}[c]{8cm}
\fontsize{33}{33}\nwpulogo 西北工业大学
\end{minipage}
\end{center}
\vskip 0.7cm
\chuhao\song {\bfseries 本科毕业设计论文}
\vskip 5cm
{
\sanhao\hei 题~~目 \hspace{0.2cm}\coverunderline[12.5cm]{基于强化学习的轮式机器人决策算法研究}
}
\vskip 2cm
{
\sihao\song 专业名称\coverunderline[7cm]{软件工程}
\vskip 0.7cm
\sihao\song 学生姓名\coverunderline[7cm]{韩~~萧~~阳}
\vskip 0.7cm
\sihao\song 指导教师\coverunderline[7cm]{史~~豪~~斌}
\vskip 0.7cm
\sihao\song 毕业时间\coverunderline[7cm]{2019.07}
\vfill
}
\end{center}
\end{titlepage}

\song \normalsize


%授权
%\include{preface/authorization}

\setcounter{page}{1}

%中文摘要
\include{preface/c_abstract}

%英文摘要
%%%%%%%%%%%%%%%%%%%%%%%%%%%%%%%%%%%%%%%%%%%%%%%%%%%%%%%%%%%%%%%%%%%%%%%%%
%
%   LaTeX File for Doctor (Master) Thesis of Tsinghua University
%   LaTeX + CJK     清华大学博士(硕士)论文模板
%   Based on Wang Tianshu's Template for XJTU
%   Version: 1.00
%   Last Update: 2003-09-12
%
%%%%%%%%%%%%%%%%%%%%%%%%%%%%%%%%%%%%%%%%%%%%%%%%%%%%%%%%%%%%%%%%%%%%%%%%%
%   Copyright 2002-2003  by  Lei Wang (BaconChina)       (bcpub@sina.com)
%%%%%%%%%%%%%%%%%%%%%%%%%%%%%%%%%%%%%%%%%%%%%%%%%%%%%%%%%%%%%%%%%%%%%%%%%

%%%%%%%%%%%%%%%%%%%%%%%%%%%%%%%%%%%%%%%%%%%%%%%%%%%%%%%%%%%%%%%%%%%%%%%%%
%
%   LaTeX File for xi'an Jiao Tong University
%
%%%%%%%%%%%%%%%%%%%%%%%%%%%%%%%%%%%%%%%%%%%%%%%%%%%%%%%%%%%%%%%%%%%%%%%%%
%   Copyright 2001  by  Wang Tianshu    (tswang@asia.com)
%%%%%%%%%%%%%%%%%%%%%%%%%%%%%%%%%%%%%%%%%%%%%%%%%%%%%%%%%%%%%%%%%%%%%%%%%
\renewcommand{\baselinestretch}{1.5}
\fontsize{12pt}{13pt}\selectfont
\setmainfont{Times New Roman}
\chapter*{\bf{ABSTRACT}}
\markboth{英~文~摘~要}{英~文~摘~要}
As a recently-recognized learning model, reinforcement learning learns through the interaction between the agent and the environment to change the action response of the agent to obtain the optimal policy. Deep neural networks provide rich representations that can enable reinforcement learning algorithms to perform effectively. Reinforcement learning methods have been applied to range of robotic tasks. However, for complex control tasks and decision system of wheeled robots in the real world environment, deep reinforcement learning typically requires significant additional work to ensure data validity and sample utilization efficiency.

In this paper, we give a brief introduction to wheeled robot decision system and reinforcement learning algorithm. We describe the basic principles of deep reinforcement learning based on value function and policy search. In the meanwhile, we elaborate the realization and training process of Deep Q-learning and Deep Deterministic Policy Gradient. In addition, we propose a method of applying multi-agent asynchronous training to solve the problem of wheeled robot delay and sampling efficiency in practical applications.

Based on the robot operating system ROS the the simulation platform Gazebo, we build a wheeled robot physics simulation environment, in which we train DQN, DDPG and multi-agent asynchronous DDPG and verify their performance and effects. We explore the adaptability of the above algorithm by using the wheeled robot agent to fight against enemy with human design DFA in the real environment.

\vspace{1em}
\noindent {\textbf{Key Words:}} \quad reinforcement learning, ROS, wheeled robot, decision system

%目录
\renewcommand{\baselinestretch}{1.25}
\fontsize{12pt}{12pt}\selectfont

\tableofcontents

%符号对照表
%\include{preface/denotation}

\mainmatter

\renewcommand{\baselinestretch}{1.5}

% 对应于小四的标准字号是 12pt
% 可以在正文中用此命令修改所需要字体的的大小
%\fontsize{12pt}{13pt}\selectfont
\xiaosi\song


%--- body --------------------------

%正文章节
\chapter{绪论}\label{preface}

\section{研究背景}
机器人技术是机械、电子、控制、计算机、人工智能等多学科交叉的领域。进入21世纪以来,国内外对机器人技术的发展越来越重视,机器人技术被认为是对未来新兴产业发展具有重要意义的高新技术之一\cite{1}。机器人的研发、制造与应用是衡量一个国家科技创新和高端制造业水平的重要标志。

机器人技术的研究和应用已经从传统的工业领域快速扩展到其他领域,如医疗健康、家政服务、外形探索、勘测勘探等。无论是传统的工业领域还是其他领域,对机器人性能要求的不断提高,使机器人必须面对更极端的环境、完成更复杂的任务。

许多国家加大对机器人技术的研究投入,并将其作为未来新兴产业寄予厚望,是未来高技术、新兴产业竞争的制高点,对于国家经济发展和国防建设具有重要意义。

近年来,我国在国家自然科学基金、863计划以及国家科技重大专项等规划中对机器人技术给予了极高的关注度。国际上,美国启动了“美国国家机器人计划”\cite{2}。欧盟在第七框架计划(FP7) 中规划了“认知系统与机器人技术”研究。日本制定了机器人技术长期发展战略。韩国制定了“智能机器人基本计划”。

\subsection{轮式机器人决策系统研究现状}
轮式移动机器人主要有智能轮椅、导游机器人、野外侦察机器人、大型智能车辆等。其定位、运动规划、自主控制、服务作业等技术和方法也得到广泛研究。随着人工智能、计算机网络技术、传感器技术等新技术的飞速发展,以及工业程度的不断提高,轮式机器人能够更好的服务社会。

目前对于轮式机器人的研究工作,主要集中在路径规划方法上。路径规划是轮式机器人研究领域的关键技术之一,旨在规划一条从起点到目标点的无碰撞路径,同时优化性能指标如距离、时间或者能耗,其中距离是最常采用的方法\cite{3}。

本文着重于在可靠路径规划算法的基础上,即在一台拥有可靠定位、避障和路径规划的机器人上,思考和探索一种能够适应复杂环境做出自主决策的轮式机器人决策系统。目前,模糊逻辑\cite{4}、决策树\cite{5}、状态机、遗传算法\cite{6}、神经网络\cite{7}等都是较为成功有效的轮式机器人决策方法。但这些方法通常需要假设完整的环境信息,然而,在大量的实际应用中需要智能体具有适应不确定环境的能力。因此,如何提高机器人路径规划的自学能力和自适应性成为当前研究的关键技术。

强化学习(Reinforcement Learning, RL)方法通过智能体与位置环境交互,并尝试动作选择使累积回报最大,该方法通常运用马尔可夫决策过程(Markov Decision Processes, MDP)进行环境建模。马尔可夫决策过程模型主要针对理想情况下的单智能体系统,智能体环境的不确定性也可由部分可观测马尔可夫决策过程(Partially Observable Markov Decision Processes, POMDP)进行描述。强化学习算法不需要给定任何状态下的指导信号,只通过智能体与环境交互进行学习并优化控制参数,在先验信息较少的复杂优化决策问题中具有广阔的应用前景。

\subsection{强化学习研究现状}
人工智能的一个首要目标就是生成能够与环境交互,通过尝试与错误学习以优化自身行为的全自主的智能体。创造一个能够有效学习的人工智能系统一直以来是一个长期的挑战,从能够感知和与周边环境进行交互的机器人到与自然语言与多媒体交互的基于软件的智能体。强化学习是一种经验驱动的全自主学习的数学方法框架\cite{8}。

强化学习的目标是需要学习一种策略,即当智能体agent处于一种状态state,做出一个动作action的决策。如果我们将动作看作对状态的标签,强化学习就可以类比监督学习,这样策略就相当于一个分类器或者回归器。主要的区别在于强化学习的数据往往需要通过尝试、和环境进行交互获得。算法则根据环境给予的反馈来调整策略。

强化学习的任务通常使用马尔可夫决策过程描述。智能体agent处于一个环境中,每个状态state为agent对环境的感知。当智能体agent执行一个动作后,环境会按照概率转移到另一个状态;同时,环境会根据奖励函数给予智能体agent一个反馈,通常是奖励reward。综合而言,强化学习主要包含四个要素:状态state、动作action、转移概率P以及奖励函数reward。

在过去,人工智能通过强化学习达成了许多成就。然而,先前的方法缺乏可泛化性并且只能在定义在相当低维空间的问题有效。随着深度神经网络的广泛使用,函数逼近和特征学习这两大法宝使我们不断克服这些问题。

深度学习的优势在机器学习的许多领域都有重大作用,显著地提升了在经典任务上的表现,例如目标检测,文本识别和语言翻译\cite{9}。深度学习最重要的特征就是深度神经网络可以自动地从图像、文本和声音等高维数据中抽象出简洁的低维特征。通过在深度神经网络中设计启发式的偏差,尤其是层级表示,机器学习使用者在解决维度灾难方面做出了有效的进步\cite{10}。随着使用深度学习算法的强化学习算法,深度强化学习算法的使用,深度学习也同样地加速了强化学习的进步。

深度学习使强化学习能够泛化应用到先前极为棘手的决策问题,比如高维状态-动作空间。在最近的强化学习工作中有三项工作的成功极为突出。

首先是深度强化学习的革命的临门一脚,能够以人类水平从图像像素级别学习游玩雅达利2600个电子游戏的智能体\cite{11}。通过为强化学习中函数逼近技术的不稳定性提供解决方案,这项工作首次令人信服地证明了强化学习智能体可以仅基于奖励信号在原始的高位观察结果上进行训练。

第二个突出的成功是开发了混合深度强化学习系统AlphaGo\cite{12},它在击败了围棋领域的人类世界冠军。与主导围棋系统的人工设计的下棋策略不同,AlphaGo是由使用监督学习和强化学习训练的神经网络组成,并结合传统的启发式搜索算法,即蒙特卡洛搜索算法。

最新的令人惊讶的工作是由OpenAI基于Dota 2应用场景开发的通用AI系统Open Five,它通过学习团队合作、长期规划和隐藏信息,开始捕捉到真实世界复杂性和连续性,并在5V5的Dota 2游戏中击败了人类顶尖的职业选手。OpenAI Five表明,当前的深度强化学习可以实现大规模的长期规划。

深度强化学习算法已经应用与广泛的问题,例如机器人技术,其中一些机器人可以直接从现实世界的摄像机输入学习控制策略\cite{13}\cite{14},而取代了人工设计的控制区或者从机器人状态的低维特征中学习。为了向更强大的智能体迈进。深度强化学习已被用于创建可以进行元学习的智能体\cite{15}\cite{16},允许模型推广到他们从未见过的复杂视觉环境。

虽然电子游戏是一个有趣的挑战,但是学习如何玩雅达利或者Dota电子游戏并不是深度强化学习的最终目标。深度强化学习背后的驱动力之一是创建能够学习如何适应现实世界的智能系统。从调度管理到装载物品,深度强化学习可以增加能够被自主学习的物理任务的数量。但是深度强化学习不知预测,因为强化学习是通过反复验证来解决优化问题的一般方法。从设计最先进的机器翻译模型到构建新的优化功能,深度强化学习已经被用于处理各种机器学习任务。并且,与深度学习在机器人学习的许多分支中的应用相同,在未来深度强化学习将是构建通用人工智能系统的重要组成部分。

\section{研究内容}
本文将使用轮式机器人中应用最广泛的操作系统ROS作为研究平台。ROS因具有完备的跨平台消息转递机制和进程处理能力而广泛应用于机器人研究中。Gazebo是ROS平台上一种功能强大的仿真环境模拟平台,本文使用Gazebo作为仿真和模拟实验的主要平台。Tensorflow作为应用最广泛的开源深度学习框架之一,能够使开发者迅速构建深度学神经网络模型,研究深度学习算法,本文采用Tensorflow作为搭建深度神经网络的基础框架。OpenAI gym 是一个应用于开发搭建和比较研究强化学习的开源工具组件,它提供了简单易用的强化学习环境搭建框架接口和模板。本文使用OpenAI gym 强化学习环境接口来封装ROS中轮式机器人感知、通信和控制等功能,使环境与全自主轮式机器人决策系统解耦,从而能够更加方便地在仿真环境与真实环境中切换。

本文的主要研究对象是全自主的轮式机器人,通过在轮式机器人上应用深度强化学习模型,期望使其具有在复杂真实环境场景下自主与敌方机器人作战的能力,即能够自主巡航、索敌、追踪和攻击。本文的主要工作有如下几个方面:
\vspace{-10pt}
\begin{enumerate}
	\item 深入分析目前主流的强化学习模型,解析其理论本质。
	\item 搭建仿真平台,模拟在复杂条件下全自主轮式机器人的决策过程。
	\item 探索深度强化学习、逆强化学习和模仿学习等多种模型与方法,以在轮式机器人上实现决策功能。
	\item 在现有理论基础上,针对轮式机器人决策问题,对网络结构、奖励函数和训练方法上做出改进。
\end{enumerate}

\section{章节安排}
本文的章节安排如下:

第\ref{preface}章为绪论,介绍了轮式机器人决策系统的概况,研究意义,给出了本文的设计方法,最后介绍了本文的研究目的,内容及结构。

第\ref{introduction}章概述轮式机器人决策系统概述。定义了本文探究的问题边界,概括介绍了轮式机器人控制系统架构,定义了轮式机器人决策系统和接口。

第\ref{rl}章详细分析了强化学习算法研究与分析,阐述了本文实验中主要使用的三种深度强化学习方法,DQN、DDPG以及多智能体异步DDPG。

第\ref{experiment}章详细分析了实验数据采集与分析,分析了在仿真平台与真实环境下的训练特点与挑战,展示了实验数据与结果。

第\ref{conclusion}章总结全文,并展望未来的研究工作。


\chapter{轮式机器人决策系统}\label{introduction}
\section{问题定义}
本文研究的轮式机器人决策问题是以ICRA DJI 机甲大师人工智能挑战赛(ICRA DJI RoboMaster AI Challenge)为基础的。该问题的定义为:在给定的室内有边界的场地内,由敌我双方各使用两台搭载有传感器与弹丸发射机构的轮式机器人在场地内进行对战。挑战赛比赛场地,是一个长为8 米、宽为5米的长方形区域,主要包含启动区、补给区、防御加成区、障碍块区和保护围挡区,如图\ref{field}所示。

\begin{figure}[htb]
  \centering
  % Requires \usepackage{graphicx}
  \includegraphics[width=\textwidth]{figures/field.png}
  \caption{挑战赛场地}\label{field}
\end{figure}

\begin{figure}[htb]
  \centering
  % Requires \usepackage{graphicx}
  \includegraphics[width=\textwidth]{figures/robot.png}
  \caption{轮式机器人与装甲模块}\label{robot}
\end{figure}


敌我双方一到两台轮式机器人分别从各自图\ref{field}的B启动区启动,在挑战赛场地内进行全自主的对抗。在挑战赛场地内,设置诸多不可移动的灰色障碍物如图\ref{field}的C障碍块区所示。同时设置了A防御加成区,轮式机器人在这一区域中停留超过5s后即可获得伤害减半的buff持续时间30s。在D补给区己方轮式机器人可以获得弹药补给。

轮式机器人可以搭载激光雷达、摄像头、UWB、超声波等多种传感器。轮式机器人四周各装有装甲模块。装甲模块装有红蓝两种色彩的LED用以区别敌我,同时装有压力传感器以检测车身是否被击中,并计算剩余血量。装甲模块如图\ref{robot}所示。

\section{决策系统架构}
\subsection{架构概述}\label{system}
在应用与实践中,我们总结形成一套基于ROS的轮式机器人的控制软件系统。该架构包括该架构包括驱动模组、感知模组、规划模组、控制模组和决策模组。

驱动模组通过选取适合的驱动软件包调用摄像头、雷达、声呐、UWB等传感器数据以获取轮式机器人实时环境信息;通过解析步兵战车通信协议,构造串口驱动类,实现对轮式机器人的控制。感知模组通过雷达、声呐、UWB等传感器信息实现地图构建、实时定位;通过轻量级深度神经网络:SSD-MoblieNets达到了在Jetson TX2开发组件上的实时目标定位与追踪。规划模组主要通过优化Navigation功能包,实现了轻量级的路径规划器,使轮式机器人在有限的计算资源下,完成了全自主巡航功能。控制模组使用了离散式增量PID控制,其结合经典控制理论与SIMULINK仿真技术,实现了对目标物的低超调高速自动追踪。

\subsection{模组通信}
我们使用ROS的Publish/Subscribe机制实现各个模组之间的通信。各个模组的数据流如图\ref{dataflow}所示。各个ROS消息结点的消息传递关系如图\ref{topic}所示。

驱动模组负责将轮式机器人状态、比赛进度、电机里程计数据、摄像头图像、激光雷达云点阵、UWB定位等数据推送给感知模组。感知模组使用目标检测、物体识别与实时定位算法,确定当前己方轮式机器人与敌方轮式机器人在场地中的位置等状态信息。决策系统根据状态信息,将决策己方轮式机器人做出的反应动作,即目标位置与姿态,并将其推送给规划模组以经行路径规划。规划模组根据目标位置与姿态使用导航算法得出底盘与云台运动的速度等数据。这些数据由控制模组处理后,通过驱动模组中的串口驱动发送给下位机执行。

\begin{figure}[htb]
  \centering
  % Requires \usepackage{graphicx}
  \includegraphics[width=\textwidth]{figures/dataflow.png}
  \caption{模组通信数据流}\label{dataflow}
\end{figure}

\begin{figure}[htb]
  \centering
  % Requires \usepackage{graphicx}
  \includegraphics[width=\textwidth]{figures/function.png}
  \caption{模组功能}\label{function}
\end{figure}

\begin{figure}[htb]
  \centering
  % Requires \usepackage{graphicx}
  \includegraphics[width=\textwidth]{figures/topic.png}
  \caption{ROS消息结点关系}\label{topic}
\end{figure}

\subsection{模组功能}
总体而言,决策系统主要依赖改软件系统的两大功能:导航和攻击,如图\ref{function}所示。导航功能主要负责在挑战赛环境下自主避障、路径规划。攻击功能主要负责实时检测、追踪和打击敌方目标。决策系统建立在在可靠的导航与攻击功能上,实际上负责协调导航与攻击功能模块的交互。决策系统可以视为模组功能沟通与协调的结点,通过接收感知信息以合理地调用各个模组功能。

\section{决策系统设计}
在ROS平台上,决策系统实际上是一个消息处理结点,它通过处理轮式机器人的感知信息,向轮式机器人发布控制命令,已协调调用各个模组的功能,从而能够实现轮式机器人全自主的巡航、索敌、追踪与打击,如图\ref{decision} 所示。
\begin{figure}[htb]
  \centering
  % Requires \usepackage{graphicx}
  \includegraphics[width=\textwidth]{figures/decision.png}
  \caption{决策结点消息交互接口}\label{decision}
\end{figure}

在决策系统设计时,考虑到可扩展性、在虚拟环境与真实环境下的适应性、与深度强化学习算法的兼容性,我们使用OpenAI gym标准接口封装了轮式机器人感知、通信与控制功能,构建一个环境适配器类env,从而实现了智能体agent与环境env的交互。智能体agent通过状态state、动作action和奖励reward与环境env进行交互,从而进行训练,如图\ref{agentenv} 所示。
\begin{figure}[htb]
  \centering
  % Requires \usepackage{graphicx}
  \includegraphics[width=\textwidth]{figures/agentenv.png}
  \caption{轮式机器人智能体与环境交互接口}\label{agentenv}
\end{figure}

按照OpenAI gym标准,我们将复杂的轮式机器人感知与控制抽象为智能体agent和环境env之间交互的三个函数:
\vspace{-10pt}
\begin{enumerate}
	\item 感知环境,$\text {env.observe}$。同步消息机制。输入为空,输出为轮式机器人智能体当前在环境中感知的状态。
	\item 动作决策,$\text {agent.choose action}$。同步消息机制。输入为轮式机器人智能体当前状态,输出为轮式机器人智能体自主做出的动作响应决策。
	\item 执行动作,$\text {env.step}$。同步消息机制。输入为轮式机器人智能体做出的动作响应,输出为环境反馈给轮式机器人智能体的奖励或惩罚和下一步。
\end{enumerate}

应用适配器模式的思想,将智能体agent做出动作决策与环境env交互解耦。环境env隐藏了决策系统结点与ROS目标识别、定位、移动控制等其他消息结点的交互细节,使智能体agent 只关心与环境env的交互。

环境env无需关心智能体agent的实现细节,通过继承env,可以是智能体在仿真轮式机器人环境与真是轮式机器人环境之间切换,从而提高了工作效率,减少了实验好耗时。

同样的,智能体agent无需关心环境env与轮式机器人的交互细节,通常来说,智能体agent只需要实现输入一个状态向量或矩阵$s$ 时,输出一个动作响应向量或矩阵$a$,就能够与环境env进行交互。因此,智能体agent可以在不同的强化学习算法、不同的神经网络模型之间灵活切换,甚至可以替换为非强化学习模型,如确定性有限状态机和行为树。

\subsection{确定性有限状态机}\label{dfa}
决策系统上我们使用人工设计的确定性有限状态机作为本文实验的基线,通过对所有可能出现的情况与相对应的行为来构建有限状态机,其中状态图如图\ref{state}所示,活动图如图\ref{activity} 所示,我们成功地对轮式机器人的行为策略进行规划,并在实际实践中取得良好的效果。

\begin{figure}[h]
  \centering
  % Requires \usepackage{graphicx}
  \includegraphics[width=\textwidth]{figures/state.png}
  \caption{轮式机器人确定性有限状态机}\label{state}
\end{figure}

\begin{figure}[h]
  \centering
  % Requires \usepackage{graphicx}
  \includegraphics[width=\textwidth]{figures/activity.png}
  \caption{轮式机器人活动图}\label{activity}
\end{figure}

\include{body/rl}
\include{body/experiment}

% 结论
%%%%%%%%%%%%%%%%%%%%%%%%%%%%%%%%%%%%%%%%%%%%%%%%%%%%%%%%%%%%%%%%%%%%%%%%%
%
%   LaTeX File for Doctor (Master) Thesis of Tsinghua University
%   LaTeX + CJK     清华大学博士(硕士)论文模板
%   Based on Wang Tianshu's Template for XJTU
%	Version: 1.00
%   Last Update: 2003-09-12
%
%%%%%%%%%%%%%%%%%%%%%%%%%%%%%%%%%%%%%%%%%%%%%%%%%%%%%%%%%%%%%%%%%%%%%%%%%
%   Copyright 2002-2003  by  Lei Wang (BaconChina)       (bcpub@sina.com)
%%%%%%%%%%%%%%%%%%%%%%%%%%%%%%%%%%%%%%%%%%%%%%%%%%%%%%%%%%%%%%%%%%%%%%%%%

\renewcommand{\baselinestretch}{1.5}
\fontsize{12pt}{13pt}\selectfont

\chapter{全文总结}\label{conclusion}
\markboth{全文总结}{全文总结}
%\addcontentsline{toc}{chapter}{\hei 总结与展望}
\section{全文总结}
深度强化学习已经准备好在AI领域掀起一场革命。深度强化学习向建立对于真实世界具有高水平理解能力的全自主系统迈出了坚实的一步。当前,深度学习使强化学习有能力处理先前极为棘手的问题。本文通过应用深度强化学习在轮式机器人上,以实现轮式机器人的全自主战斗决策。主要做的工作如下:
\vspace{-10pt}
\begin{enumerate}
  \item 基于ROS,构建了轮式机器人的决策系统。通过将智能体agent部分与环境env部分解耦,实现了决策系统的模块化。
  \item 简述了强化学习的基本原理,介绍了基于值函数与策略搜索的强化学习基本算法。
  \item 基于Gazebo,搭建了轮式机器人仿真平台。
  \item 提出了多智能体异步的训练方法,解决了真实环境下的时延问题和采样速度问题。
  \item 实现并检验了DQN、DDPG和多智能体异步DDPG在仿真环境和真实环境下的训练和表现。
\end{enumerate}

\section{对未来工作的展望}
根据本文的分析,可以发现深度强化学习已经广泛应用于机器人控制任务,从简单运动到复杂操作和全自主移动机器人控制。然而,强化学习在真实世界中的实际应用通常需要超出学习算法本身的大量额外工程:
\vspace{-10pt}
\begin{enumerate}
  \item 机器人无法感知到真实世界的全貌,即机器人只能部分感知当前状态。
  \item 真实世界往往是奖励稀疏的,如何确定奖励函数是一个棘手的问题。
  \item 机器人在真实环境下存在着感知、通信和控制时延,这还会使得机器人无法。
\end{enumerate}

在未来的工作中,应该着重与解决这些问题。建立基于部分部分可观测马尔可夫决策过程的模型和算法;应该设计基于状态-动作对概率分布的奖励函数,使连续的状态-动作空间都能收到奖励或者惩罚信号;使用并发和分布式的方法,克服机器人在真实世界中训练的时延。



%参考文献
\wuhao

%\bibliographystyle{unsrt}
\bibliographystyle{GBT7714-2005NLang}
%\ifpdf \phantomsection \fi

%\addcontentsline{toc}{chapter}{\hei 参考文献}
%\addtolength{\itemsep}{-0.8 em} % 缩小参考文献间的垂直间距, 在bibtex下无效
%\bibliography{reference/reference}

% 致谢

\clearpage
\phantomsection
\addcontentsline{toc}{chapter}{{\bf 参考文献}}
%\addtolength{\itemsep}{-0.5pt}  %修改参考文献间距

{\renewcommand\baselinestretch{1.25}\selectfont


\begin{thebibliography}{999}
\addtolength{\itemsep}{-1.05 em} % 缩小参考文献间的垂直间距
\setlength{\itemsep}{1pt}
%\renewcommand{\baselinestretch}{1.05}
\bibitem{1}
徐扬生. 智能机器人引领高新技术发展[J]. 企业科协, 2010 (9): 28-31.
\bibitem{2}
Christensen H I, Batzinger T, Bekris K, et al. A roadmap for us robotics: from internet to robotics[J]. Computing Community Consortium, 2009, 44.
\bibitem{3}
Salichs M A, Moreno L. Navigation of mobile robots: open questions[J]. Robotica, 2000, 18(3): 227-234.
\bibitem{4}
Beom H R, Cho H S. A sensor-based navigation for a mobile robot using fuzzy logic and reinforcement learning[J]. IEEE transactions on Systems, Man, and Cybernetics, 1995, 25(3): 464-477.
\bibitem{5}
Pandey A, Sonkar R K, Pandey K K, et al. Path planning navigation of mobile robot with obstacles avoidance using fuzzy logic controller[C]//2014 IEEE 8th International Conference on Intelligent Systems and Control (ISCO). IEEE, 2014: 39-41.
\bibitem{6}
Wang C, Soh Y C, Wang H, et al. A hierarchical genetic algorithm for path planning in a static environment with obstacles[C]//IEEE CCECE2002. Canadian Conference on Electrical and Computer Engineering. Conference Proceedings (Cat. No. 02CH37373). IEEE, 2002, 3: 1652-1657.
\bibitem{7}
Pandey A, Sonkar R K, Pandey K K, et al. Path planning navigation of mobile robot with obstacles avoidance using fuzzy logic controller[C]//2014 IEEE 8th International Conference on Intelligent Systems and Control (ISCO). IEEE, 2014: 39-41.
\bibitem{8}
Sutton R S, Barto A G. Introduction to reinforcement learning[M]. Cambridge: MIT press, 1998.
\bibitem{9}
LeCun Y. Yoshua Bengio, and Geoffrey Hinton[J]. Deep learning. nature, 2015, 521(7553): 436-444.
\bibitem{10}
Bengio Y, Courville A, Vincent P. Representation learning: A review and new perspectives[J]. IEEE transactions on pattern analysis and machine intelligence, 2013, 35(8): 1798-1828.
\bibitem{11}
Mnih V, Kavukcuoglu K, Silver D, et al. Human-level control through deep reinforcement learning[J]. Nature, 2015, 518(7540): 529.
\bibitem{12}
Silver D, Huang A, Maddison C J, et al. Mastering the game of Go with deep neural networks and tree search[J]. nature, 2016, 529(7587): 484.
\bibitem{13}
Levine S, Finn C, Darrell T, et al. End-to-end training of deep visuomotor policies[J]. The Journal of Machine Learning Research, 2016, 17(1): 1334-1373.
\bibitem{14}
Levine S, Pastor P, Krizhevsky A, et al. Learning hand-eye coordination for robotic grasping with deep learning and large-scale data collection[J]. The International Journal of Robotics Research, 2018, 37(4-5): 421-436.
\bibitem{15}
Duan Y, Schulman J, Chen X, et al. RL $^ 2$: Fast Reinforcement Learning via Slow Reinforcement Learning[J]. arXiv preprint arXiv:1611.02779, 2016.
\bibitem{16}
Wang J X, Kurth-Nelson Z, Kumaran D, et al. Prefrontal cortex as a meta-reinforcement learning system[J]. Nature neuroscience, 2018, 21(6): 860.
\bibitem{17}
Vinyals O, Ewalds T, Bartunov S, et al. Starcraft ii: A new challenge for reinforcement learning[J]. arXiv preprint arXiv:1708.04782, 2017.
\bibitem{18}
Kaelbling L P, Littman M L, Cassandra A R. Planning and acting in partially observable stochastic domains[J]. Artificial intelligence, 1998, 101(1-2): 99-134.
\bibitem{19}
Watkins C J C H, Dayan P. Q-learning[J]. Machine learning, 1992, 8(3-4): 279-292.
\bibitem{20}
Rummery G A, Niranjan M. On-line Q-learning using connectionist systems[M]. Cambridge, England: University of Cambridge, Department of Engineering, 1994.
\bibitem{21}
Koutník J, Cuccu G, Schmidhuber J, et al. Evolving large-scale neural networks for vision-based reinforcement learning[C]//Proceedings of the 15th annual conference on Genetic and evolutionary computation. ACM, 2013: 1061-1068.
\bibitem{22}
Deisenroth M P, Neumann G, Peters J. A survey on policy search for robotics[J]. Foundations and Trends® in Robotics, 2013, 2(1–2): 1-142.
\bibitem{23}
Williams R J. Simple statistical gradient-following algorithms for connectionist reinforcement learning[J]. Machine learning, 1992, 8(3-4): 229-256.
\bibitem{24}
Konda V R, Tsitsiklis J N. Onactor-critic algorithms[J]. SIAM journal on Control and Optimization, 2003, 42(4): 1143-1166.
\bibitem{25}
Schulman J, Moritz P, Levine S, et al. High-dimensional continuous control using generalized advantage estimation[J]. arXiv preprint arXiv:1506.02438, 2015.
\bibitem{26}
Lillicrap T P, Hunt J J, Pritzel A, et al. Continuous control with deep reinforcement learning[J]. arXiv preprint arXiv:1509.02971, 2015.
\bibitem{27}
Gu S, Holly E, Lillicrap T, et al. Deep reinforcement learning for robotic manipulation with asynchronous off-policy updates[C]//2017 IEEE international conference on robotics and automation (ICRA). IEEE, 2017: 3389-3396.
\bibitem{28}
Van Hasselt H, Guez A, Silver D. Deep reinforcement learning with double q-learning[C]//Thirtieth AAAI Conference on Artificial Intelligence. 2016.
\bibitem{29}
Wang Z, Schaul T, Hessel M, et al. Dueling network architectures for deep reinforcement learning[J]. arXiv preprint arXiv:1511.06581, 2015.
\bibitem{30}
Schaul T, Quan J, Antonoglou I, et al. Prioritized experience replay[J]. arXiv preprint arXiv:1511.05952, 2015.
\end{thebibliography}

%%%%%%%%%%%%%%%%%%%%%%%%%%%%%%%%%%%%%%%%%%%%%%%%%%%%%%%%%%%%%%%%%%%%%%%%%
%
%   LaTeX File for Doctor (Master) Thesis of Tsinghua University
%   LaTeX + CJK     清华大学博士(硕士)论文模板
%   Based on Wang Tianshu's Template for XJTU
%   Version: 1.00
%   Last Update: 2003-09-12
%
%%%%%%%%%%%%%%%%%%%%%%%%%%%%%%%%%%%%%%%%%%%%%%%%%%%%%%%%%%%%%%%%%%%%%%%%%
%   Copyright 2002-2003  by  Lei Wang (BaconChina)       (bcpub@sina.com)
%%%%%%%%%%%%%%%%%%%%%%%%%%%%%%%%%%%%%%%%%%%%%%%%%%%%%%%%%%%%%%%%%%%%%%%%%


%%%%%%%%%%%%%%%%%%%%%%%%%%%%%%%%%%%%%%%%%%%%%%%%%%%%%%%%%%%%%%%%%%%%%%%%%
%
%   LaTeX File for phd thesis of xi'an Jiao Tong University
%
%%%%%%%%%%%%%%%%%%%%%%%%%%%%%%%%%%%%%%%%%%%%%%%%%%%%%%%%%%%%%%%%%%%%%%%%%
%   Copyright 2002  by  Wang Tianshu    (tswang@asia.com)
%%%%%%%%%%%%%%%%%%%%%%%%%%%%%%%%%%%%%%%%%%%%%%%%%%%%%%%%%%%%%%%%%%%%%%%%%
\renewcommand{\baselinestretch}{1.5}
\fontsize{12pt}{13pt}\selectfont

\chapter*{致~~~~谢}
\markboth{致谢}{致谢}
\addcontentsline{toc}{chapter}{\hei 致谢}
首先要感谢我的毕业设计指导老师史豪斌副教授。论文是在史老师的悉心指导下完成的。在论文的进展中,史老师提供了实验平台和学习资料,在学业和论文写作上提出了许多宝贵意见。史老师严谨的治学态度以及丰富的实践经验将是我以后学习和工作的动力和楷模。在此谨向我的指导老师史豪斌副教授表示衷心的感谢。

同时要感谢西北工业大学竞技机器人基地的同学和队友们,是我们一起共同的努力,让我们能在ICRA的平台上进行机器人与人工智能的竞技和挑战。



\chapter*{毕业设计小结}
\markboth{毕业设计小结}{毕业设计小结}
\addcontentsline{toc}{chapter}{\hei 毕业设计小结}
这次毕业设计是对我四年本科学习的一个总结,涉及了机器人感知、控制与通信,深度神经网络与深度强化学习,这对我来说是一个全面的考验。虽然在过去的比赛中经常接触和使用仿真环境Gazebo以及机器人操作系统ROS,但是对它们的理解并不算透彻。在这次毕业设计的过程中,我又系统地学习了ROS,深刻理解了ROS消息机制。对于深度神经网络方面,我在这次的毕业设计实践中感受数学工具在开辟新算法和思路时的重要性以及扎实的打好数学基础对于计算机相关专业学习的重要性。

通过本次毕业设计,我对深度强化学习在轮式机器人上的应用有了更深入的了解,对今后研究生阶段的学习和奋斗的目标也更加明确。






%  附录

%\begin{appendix}
%    \renewcommand{\chaptername}{附录\Alph{chapter}}
%   \input{appendix/appendix.tex}
%\end{appendix}

% 发表的文章列表

%\include{appendix/publications}

\clearpage
\end{document}

%%%%%%%%%%%%%%%%%% End of the file  %%%%%%%%%%%%%%%%%%%%%%%%
